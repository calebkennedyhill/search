\section{Basics}

Let $G = (V,E)$ be a graph with $V=\{1,\dots,n\}$.
Define $\unit$ to be the $n\times1$ matrix (i.e., column vector) with $1$ in every slot.
Let $A$ be the adjacency matrix of $G$.

\begin{proposition}
    $D_{i,i}$ is the number of neighbors of $i$. 
\end{proposition}
\begin{proof}
    
\end{proof}

\begin{definition}
    $L\coloneq D-A$
\end{definition}

\begin{proposition}
    $L_{i,j} = \begin{cases}
        D_{i,i} & i=j \\
        -1 & \\
        0 & \\
    \end{cases}$
\end{proposition}
\begin{proof}
    
\end{proof}





\section{Search}

There might be a name for this somewhere.

\begin{definition}
    Call a graph {\bf local} if its adjacency matrix is block diagonal.

    \red{This is the WRONG definition.}
\end{definition}

It seems like a local graph would lend itself more to parallel BFS.

\begin{proposition}
    Suppose a real symmetric matrix $B$ is block-diagonal 
    with blocks $B_1,\dots,B_N$.
    Then the $B_i$ share a positive eigenvalue.
    Let $\lambda$ be such a shared eigenvalue, and suppose $v_1,\dots,v_N$ are the
    corresponding eigenvectors.
    Then $(v_1^T, \dots, v_N^T)^T$ is an eigenvector for $B$ with eigenvalue $\lambda$.
\end{proposition}
\begin{proof}
    
\end{proof}
