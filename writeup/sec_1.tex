\section{Basics}

Let $G = (V,E)$ be a graph with $V=\{1,\dots,n\}$.
Define $\unit$ to be the $n\times1$ matrix (i.e., column vector) with $1$ in every slot.
Let $A$ be the adjacency matrix of $G$.

\begin{proposition}
    $D_{i,i}$ is the number of neighbors of $i$. 
\end{proposition}
\begin{proof}
    
\end{proof}

\begin{definition}
    $L\coloneq D-A$
\end{definition}

\begin{proposition}
    $L_{i,j} = \begin{cases}
        D_{i,i} & i=j \\
        -1 & (i,j)\in E \\
        0 & \text{else} \\
    \end{cases}$
\end{proposition}
\begin{proof}
    
\end{proof}





\section{Search}

I have at least two questions:
\begin{enumerate}
    \item How does the fact that nodes have 2, 3, or 4 neighbors manifest in the adjacency matrix?
    \item Can we exploit this structure to parallelize exploration?
\end{enumerate}


\subsection{Structure of the graph}

The first goal here is to write down the adjacency matrix for the 100 
or so nodes surrounding the $3\times3$ solved state. 
This is to start addressing question (1) above.
See \verb|explore_puzzle_space.py| for progress.

We'll be considering, e.g., a $3\times3$ puzzle configuration as a permutation of the 
set 
\[
    \{1,2,3,4,5,6,7,8,0\}.
\]  
There are some particularities about where 0 goes in the null/solved permutation,
but that shouldn't be too important now.
Permutations will be denoted by $\sigma$.

Let $\sigma_0,\dots,\sigma_{99}$ be the configurations of the 100 nodes surrounding the solved state.
Let $\lambda(\sigma)$ denote the \hyperlink{https://en.wikipedia.org/wiki/Lehmer_code}{Lehmer encoding}
of the permutation $\sigma$.
So $\lambda(\sigma) \in \{ 0,\dots,99\}$.
Let $i_0,\dots,i_{99}$ be such that $\lambda(\sigma_{i_0}) < \cdots < \lambda(\sigma_{i_{99}})$,
and define $f$ by $f(\lambda(\sigma_{i_k})) \coloneq k$.

Now, whenever a new node $n$ is discovered from parent $p$, keep track of number $\lambda(n.\sigma)$
as well as the pair 
\[
    (\lambda(n.\sigma), \lambda(p.\sigma)).
\]

The size of the set $\{ n.\sigma \}$ will tell us how big to make the adjacency matrix,
and the pairs $(\lambda(n.\sigma), \lambda(p.\sigma))$ will tell us which entries are nonzero.